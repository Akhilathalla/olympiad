\documentclass[12pt,-letter paper]{article}
\usepackage{siunitx}
\usepackage{setspace}
\usepackage{gensymb}
\usepackage{xcolor}
\usepackage{caption}
%\usepackage{subcaption}
\doublespacing
\singlespacing
\usepackage[none]{hyphenat}
\usepackage{amssymb}
\usepackage{relsize}
\usepackage[cmex10]{amsmath}
\usepackage{mathtools}
\usepackage{amsmath}
\usepackage{commath}
\usepackage{amsthm}
\interdisplaylinepenalty=2500
%\savesymbol{iint}
\usepackage{txfonts}
%\restoresymbol{TXF}{iint}
\usepackage{wasysym}
\usepackage{amsthm}
\usepackage{mathrsfs}
\usepackage{txfonts}
\let\vec\mathbf{}
\usepackage{stfloats}
\usepackage{float}
\usepackage{cite}
\usepackage{cases}
\usepackage{subfig}
%\usepackage{xtab}
\usepackage{longtable}
\usepackage{multirow}
%\usepackage{algorithm}
\usepackage{amssymb}
%\usepackage{algpseudocode}
\usepackage{enumitem}
\usepackage{mathtools}
%\usepackage{eenrc}
%\usepackage[framemethod=tikz]{mdframed}
\usepackage{listings}
%\usepackage{listings}
\usepackage[latin1]{inputenc}
%%\usepackage{color}{   
%%\usepackage{lscape}
\usepackage{textcomp}
\usepackage{titling}
\usepackage{hyperref}
%\usepackage{fulbigskip}   
\usepackage{tikz}
\usepackage{graphicx}
\lstset{
  frame=single,
  breaklines=true
}
\let\vec\mathbf{}
\usepackage{enumitem}
\usepackage{graphicx}
\usepackage{siunitx}
\let\vec\mathbf{}
\usepackage{enumitem}
\usepackage{graphicx}
\usepackage{enumitem}
\usepackage{tfrupee}
\usepackage{amsmath}
\usepackage{amssymb}
\usepackage{mwe} % for blindtext and example-image-a in example
\usepackage{wrapfig}
\graphicspath{{figs/}}
\providecommand{\mydet}[1]{\ensuremath{\begin{vmatrix}#1\end{vmatrix}}}
\providecommand{\myvec}[1]{\ensuremath{\begin{bmatrix}#1\end{bmatrix}}}
\providecommand{\cbrak}[1]{\ensuremath{\left\{#1\right\}}}
\providecommand{\brak}[1]{\ensuremath{\left(#1\right)}}
\begin{document}
\begin{enumerate}
	\item Determine all functions $f$:$\mathbb{R}$ $\rightarrow$ $\mathbb{R}$ such that the euality
		$f(\lfloor{x}\rfloor{y})$=$f(x)\lfloor{f(y)}\rfloor$
		holds for all $x,y$ $\in$ $\mathbb{R}$.(Here $\lfloor{z}\rfloor$ denotes the greatest integer less than or equal to $z$.)
	\item Let $I$ be the incentre of triangle $ABC$ and let $\Gamma$ be its circumcircle. Let the line $AI$ intersect $\Gamma$ again at $D$. Let $E$ be a point on the arc $\overline{BDC}$ and $F$ a point on the side $BC$ such that
	$\angle{BAF}$ = $\angle{CAE}\angle\frac{1}{2}\angle{BAC}$.
		Finally, let $G$ be the midpoint of the segment $IF$. Prove that the lines $DG$ and $EI$ intersect on $\Gamma$.
	\item Let $\mathbb{N}$ be the set of positive integers. Determine all functions $g$:$\mathbb{N}$ $\rightarrow$ $\mathbb{N}$ such that
		$ \brak g \brak m +n $ $ \brak m +g \brak n $
		is a perfect square for all $m,n$ $\in$ $\mathbb{N}$.
	\item Let $P$ be a point inside the triangle $ABC$. The lines $AP$, $BP$ and $CP$ intersect the circumcircle $\Gamma$ of triangle $ABC$ again at the points $K,L$ and $M$ respectively. The tangent to $\Gamma$ at $C$ intersects the line $AB$ at $S$. Suppose that $SC$=$SP$. Prove that $MK$=$ML$.
	\item In each of six boxes $B_{1}$, $B_{2}$, $B_{3}$, $B_{4}$, $B_{5}$, $B_{6}$ there is initially one coin. There are two types of operation allowed: \\
		Type $1$: Choose a nonempty box $B_{j}$ with $1\leq{j}\leq{5}$. Remove one coin from $B_{j}$ and add two coins to $B_{j+1}$. \\
		Type {2}: Choose a nonempty box $B_{k}$ with $1\leq{k}\leq{4}$. Remove one coin from $B_{k}$ and exchange the contents of (possible empty) boxes $B_{k+1}$ and $B_{k+2}$. \\
		Determine whether there is a finite sequence of such operations that results in boxes $B_{1}$, $B_{2}$, $B_{3}$, $B_{4}$, $B_{5}$ being empty and box $B_{6}$ containing exactly $2010^{2010^{2010}}$ coins. (Note that $a^{(b^{c})}$.)
	\item Let $a_{1}$, $a_{2}$, $a_{3}$,... be a sequence of positive real numbers. Suppose that for some positive integer s, we have 
		$a_{n}=max \{a_{k}+a_{n-k} | 1\leq{k}\leq{n-1\}$
		for all $n>s$. Prove that there exist positive integers $l$ and $N$, with $l\leq{s}$ and such that $a_{n}$=$a_{l}$+$a_{n-l}$ for all $n\leq{N}$.
\end{enumerate}
\end{document}
